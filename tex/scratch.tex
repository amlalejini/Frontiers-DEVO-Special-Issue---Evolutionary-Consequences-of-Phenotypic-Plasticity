%%%%%%%%%%%%%%%%%%%%%%%%%%%%%%%%%%%%%%%%%%%%%%%%%%%%%%%%%%%%%%%%%%%%%%%%%
% Old abstract - 2021-01-13
%%%%%%%%%%%%%%%%%%%%%%%%%%%%%%%%%%%%%%%%%%%%%%%%%%%%%%%%%%%%%%%%%%%%%%%%%

% \noindent
% Fluctuating environmental conditions are ubiquitous in natural systems, and 
% the particular mechanisms that populations rely on to cope with environmental fluctuations can profoundly influence subsequent evolutionary dynamics. 
% The particular *evolved* adaptations/mechanisms
% Phenotypic plasticity is the capacity for a single genotype to alter phenotypic expression in response to environmental conditions, allowing plastic genotypes to dynamically adjust to environmental fluctuations and stabilizing populations against these changes. 
% Here, we use digital evolution to investigate how the capacity for phenotypic plasticity alters evolutionary dynamics and influences evolutionary outcomes in fluctuating environments. 
%alters evolutionary dynamics and influences the tape of life
% to explore the evolutionary consequences of phenotypic plasticity in fluctuating environments
% In digital evolution, populations of self-replicating computer programs (digital organisms) compete, mutate, and evolve in computational environments. 
%[benefit of digital evolution in context of this work?]
% We find that plasticity buffers populations against environmental fluctuations, such that plastic populations undergo fewer selective sweeps and accumulate fewer genetic changes than non-plastic populations.
% The lineages of non-plastic organisms exhibit increased phenotypic volatility, as non-plastic lineages must continuously re-adapt to environmental changes. 
% [landscape results - short-term evolvability, buttressing pleitropy]
% Additionally, we demonstrate that the buffering effect of plasticity allows plastic populations to [evolve more complex low-reward metabolic functions], whereas non-plastic populations are subjected to more deleterious hitchhiking effects]. 
% % [results are consistent with current theory/literature but have yet to be comprehensively demonstrated in an evolving system?]
% % -> move this sort of point to the conclusion?
% [Zoom out: these findings help inform/allow us to predict how populations respond to fluctuating environments].
% Talk about how in nature all environments involve some form of change, as such these findings are important to inform how populations respond to one type of changing environment involving direct environmental fluctuations

% We find that the capacity for phenotypic plasticity 
% plastic populations underwent fewer selective sweeps and accumulated genetic changes at a slower rate than non-plastic populations in the same environment. 

% such that evolutionary dynamics of plastic populations evolving in a fluctuating environment more closely resemble non-plastic populations in static environments than non-plastic populations in fluctuating environments.

% We explore how plasticity affects evolutionary dynamics/consequences. 
%, allowing a single genotype to express optimal phenotype across environments
%In the absence of reliable phenotypic plasticity, fluctuating environments may select for genotypes that average across possible environmental conditions or genotypes that rely on 

%%%%%%%%%%%%%%%%%%%%%%%%%%%%%%%pulations with adaptive phenotypic plasticity evolve more slowly than non-plastic12populations, which rely on genetic variation fromde novomutations to continuously re-adapt13to the environment. These non-plastic populations undergo more frequent selective sweeps,14accumulate many more genetic changes, and move to regions of the fitness landscape where15mutations will more frequently result in phenotypic changes. We find that phenotypic plasticity16stabilizes populations against environmental fluctuations; whereas the repeated selective sweeps17in non-plastic populations drive the loss of beneficial traits via deleterious hitchhiking. As such,18plastic populations are significantly more likely to retain novel adaptive traits than their non-plastic19counterparts. All natural environments subject populations to some form of change; our findings20suggest that the stabilizi%%%%%%%%%
% Scratch
%%%%%%%%%%%%%%%%%%%%%%%%%%%%%%%%%%%%%%%%
% inform how the evolution of phenotypic plasticity can stabilize population dynamics and can dramatically alter how the tape of life plays out. 


% how the tape of life plays forward 


% involve some form of change, as such these findings are important to inform how populations respond to one type of changing environment involving direct environmental fluctuations

% Our work suggests that the stabilizing effect of phenotypic plasticity on a population's evolutionary dynamics plays an important role in adaptive evolution in fluctuating environments. 

%plasticity's stabilizing effect on population dynamics may play an important role on the evolution and maintaince of complex features in natural populations, and suggests that 

%mechanism is population effect on population dynamics

%Our results demonstrate that the capacity for phenotypic plasticity can stabilize populations against environmental fluctuations, and as a result, plastic populations are more likely to evolve and maintain novel 

%can stabilize populations against environmental fluctuations, and as such, 


% We find that the capacity for phenotypic plasticity can stabilize populations against environmental fluctuations, and facilitate the maintenance of 


% The frequent selective sweeps driven by harsh environmental change increases the prevalence of deleterious hitchhiking effects in non-plastic populations

% We find that non-plastic populations evolve more quickly than plastic populations in fluctuating environments.
% Non-plastic populations accumulate many more genetic and phenotypic changes along their phylogenetic histories and undergo more frequent selective sweeps than plastic populations, as non-plastic populations 
% We demonstrate that phenotypic plasticity can stabilize populations against environmental fluctuations, and as a result, we find that plastic populations are more likely to retain 


% We find that plasticity can stabilize populations in the context of environmental fluctuations, and we observe that this stabilization facilitates the evolution of additional, more complex metabolic tasks relative to non-plastic populations subjected to equivalent levels of environmental fluctuations. 

% The lineages of non-plastic organisms exhibit increased phenotypic volatility, as non-plastic lineages must continuously re-adapt to environmental changes. 
%We show that non-plastic populations that must continuously re-adapt to environmental changes undergo frequent rapid selective sweeps, 
%resulting in repeated losses in genetic diversity
%deleterious hitchhiking
% Talk about how in nature all environments involve some form of change, as such these findings are important to inform how populations respond to one type of changing environment involving direct environmental fluctuations


%Phenotypic plasticity can stabilize populations against environmental fluctuations, influencing subsequent evolutionary outcomes. 
% Here, we use digital evolution to experimentally investigate how the capacity for phenotypic plasticity alters subsequent evolutionary dynamics and influences evolutionary outcomes in fluctuating environments.
% In digital evolution, populations of self-replicating computer programs (digital organisms) compete, mutate, and evolve in a computational environment.

%%%%%%%%%%%%%%%%%%%%%%%%%%%%%%%%%%%%%%%%
% Introduction
%%%%%%%%%%%%%%%%%%%%%%%%%%%%%%%%%%%%%%%%

% Background
% -- Questions from papers --
% - Long standing question => does plasticity observed within generations influence subsequent evolutionary change across generations?
% - Phenotypic plasticity slows the rate of evolutionary change because most of the phenotypic variance is environmentally induced and non-heritable.
% - Does plasticity create novel opportunities for selection or does it buffer environmental variation and dampen selection?
% - Phenotypic plasticity can stabilize a population against environmental changes?
% - Hypotheses about phenotypic plasticity
% - Challenging to test because X, Y, Z
% -- Points from papers --
% - Phenotypic plasticity is thought to influence species extinction risk: either by buffering a population against environmental changes, allowing individuals to survive outside of normal range of environmental conditions.
% % how populations and species respond to environmental changes is critical to persistence both now and in future
% - Alternatively, environmental changes that affect reliability environmental cues that influence/adjust/trigger plasticity may put species at risk/susceptibility to extinction. 
% - Plasticity may sometimes promote and may sometimes constrain adaptive evolution.
% %  - promote: plasticity might allow for the colonization of, and persistence in, extreme environments; create novel opportunities for selection
% %  - constrain: plasticity shields genotype from selection thereby slowing adaptive genetic change
% - Plasticity might slow the rate of evolutionary change because most phenotypic variance is environmentally induced and non-heritable [citations].
% - Plasticity source of cryptic variation that may be selected on and eventually canalized, driving adaptive evolution.
% - Understanding the evolutionary consequences of phenotypic plasticity is particularly important in the context of [rapid progression of global climate change] [citations]. 
% % - plasticity modulates how environmental variation influences population dynamics [reed et al., 2010]
% - However, experimentally test hypotheses/experimentally address questions difficult in natural systems because manipulating organisms' capacity for phenotypic plasticity can be impractical and is not always possible. 

% % - plasticity modulates how environmental variation influences population dynamics [reed et al., 2010]
% - However, experimentally test hypotheses/experimentally address questions difficult in natural systems because manipulating organisms' capacity for phenotypic plasticity can be impractical and is not always possible. 
% relationship between phenotypic plasticity and sebsequent evolution


% plasticity buffering against changing conditions => allowing organisms to surivive where they otherwise wouldn't => 
% populations relying on existing cues if cues change

% relationship between phenotypic plasticity and evolution => global climate change

% Fluctuating selection pressures have been shown to favor the evolution of [varied/a wide range of] mechanisms and life history strategies for coping with environmental changes, such as phenotypic plasticity or bet-hedging strategies [citations].

%Evolutionary prediction requires us to understand the mechanisms likely to evolve due to different types of fluctuations and the resulting dynamics that they produce.
% Understanding these relationships is critical to [ability to predict how evolution plays out in context of environmental change].
% Phenotypic plasticity is the capacity for a single genotype to alter phenotypic expression in response to environmental conditions [citations].
% In this work, we use digital evolution experiments to investigate how the evolution of adaptive phenotypic plasticity influences subsequent evolutionary dynamics in a fluctuating environment.


% evolutionary histories, 
% - Evolutionary dynamics
%   - plasticity stabilizes populations in fluctuating environment, buffering against changes. 
%   - Indeed, we find that the dynamics of plastic populations in fluctuating environment more closely resemble the dynamics of a population in a constant environment. 
%   - plastic populations evolve more slowly, plastic populations undergo fewer selective sweeps and we observe that lineages of plastic genotypes accumulate fewer mutations than nonplastic counterparts.
% - Consequences
%   - complex features
%     - despite slower overall evolutionary change,...
%   - deleterious hitchhiking
% hope that this study inspires further work exploring how plasticity affects dynamics

%%%%%%%%%%%%%%%%%%%%%%%%%%%%%%%%%%%%%%%%
% Methods
%%%%%%%%%%%%%%%%%%%%%%%%%%%%%%%%%%%%%%%%

%%%%%%% Experimental design

% independent replicates of evolving 
% We control the capacity for adaptive phenotypic plasticity to evolve by enabling or disabling organisms' sensory mechanisms. 
% The acclimation phase affords populations time to adapt to a common fluctuating environment.

% The evolution of adaptive phenotypic plasticity is not a prescribed outcome during the acclimation stage of the experiment (even when the requisite conditions are met [cite]); 

% @AML: this alternative lead-in paragraph describes overall experimental setup, but I'm thinking that this might be better off in the first experiment's description => allows me to be slightly more concrete/specific throughout.
% We conducted a series of experiments using Avida to investigate how the evolution of adaptive plasticity alters evolutionary dynamics and influences evolutionary outcomes in fluctuating environments.
% Each experiment is divided into two phases: an `acclimation' phase and a `measurement' phase.
% During the acclimation phase, we subject evolving populations of digital organisms to a fluctuating environment (cycling between ENV-A and ENV-B); each independent population is seeded with a common ancestor capable only of self-replication. 
% The acclimation phase affords each population time to adapt to their environment; this phase gives adaptive plasticity the opportunity to evolve in experimental treatments with the requisite conditions. 
% During the [second] phase of each experiment, we transfer the most numerous genotype (i.e., the dominant genotype) from each replicate to seed new populations.
% We compare across treatments only during this [second] experiment phase.
% The evolution of adaptive phenotypic plasticity is not a prescribed outcome during the acclimation phase for conditions in which plasticity is expected to evolve [cite]. 
% [as such, for conditions with the purpose of representing the dynamics of plastic populations, we only continue to the second phase if plasticity evolves].

% Avida is an open-source software platform that provides an instance of evolution \textit{in silico}, allowing researchers to test hypotheses that would be difficult or impossible to address in natural systems.

% The expression of an organism's genome in the context of its virtual hardware and environment [is recorded as its phenotype].
% Avida measures an organism's phenotype by tracking its interactions with its environment.
% Avida tracks each organism as it expresses its genome, measuring an organism's phenotype 
% An organism's phenotype results from the expression of its genome in the context of its virtual hardware and environment.
%The [programming/genetic/?] language that specifies/encodes/constitutes genomes in Avida is Turing Complete (i.e., is able to represent any computable function) and syntactically robust (i.e., any ordering of instructions is syntactically valid, though not necessarily meaningful).

% The genomes are made up of instructions from a computationally universal genetic language (i.e., able to represent any computable function).
% This genetic language is also syntactically robust; that is, any ordering of instructions is syntactically valid, though not necessarily meaningful.
% The instruction set used to specify genomes in Avida is Turing Complete (i.e., is able to represent any computable function) and syntactically robust (i.e., any ordering of instructions is syntactically valid, though not necessarily meaningful).


% Thus, we limit our second phase [comparisons] to

% We compare the evolutionary dynamics of evolving populations during the [second] phase of the experiment.
% This ensures that we limit our observations to populations that are well-adapted to their experimental treatment.
% Because we only continue to the [second] phase in the plasticity-enabled condition if adaptive plasticity evolves, we limit our second phase measurements to phenotypically plastic populations.  

The environment cycles between equal-length periods of ENV-A and ENV-B (as described in Section \ref{sec:methods:evolution-of-plasticity-in-avida}).
% Each of these periods persist for 100 updates; thus, populations experience a total of 1,000 full cycles of ENV-A and 1,000 full cycles of ENV-B during the acclimation phase.
% Each independent population (regardless of experimental treatment) is seeded with a common ancestor capable only of self-replication.
% The acclimation phase gives populations time to adapt to their treatment conditions, affording adaptive plasticity the opportunity to evolve \textit{de novo} in treatments with the requisite conditions; we never seed the acclimation phase with phenotypically plastic genotypes.
% At the end of the acclimation phase, we extract the dominant (i.e., most numerous) genotype from each replicate population to seed a new population for the [second, ``X'',] phase our experiment.
% % to seed a new population 
% The evolution of adaptive phenotypic plasticity is not a prescribed outcome in treatments intended to produce plastic populations; for such experimental treatments, we only continue to the [second] phase if the dominant genotype [correctly regulates which tasks it performs in ENV-A and ENV-B.]

% % TODO - work in more direct statement about plastic v non-plastic?
% The [second] phase proceeds identically to the first. %; except, we seed populations with a digital organism that is well-adapted to the particular treatment conditions (as opposed to the initial common ancestor).
% During the [second] phase, we track the full evolutionary histories of evolving populations: 
% we maintain the full phylogenies (i.e., [short definition]) of extant populations, including genomes, phenotypic information (e.g., [examples of things we track]), and mutations for post-hoc analysis.
% % @AML: feel like we need one more sentence here to give closure to the overall experimental design?

% We evolve \evolutionaryChangeRateReplicates\ independent replicate populations under each of three experimental conditions:
% (1) a plasticity-enabled treatment where digital organisms can differentiate between ENV-A and ENV-B using sensory instructions (adaptive plasticity can evolve),
% (2) a non-plastic treatment where sensory instructions are disabled (i.e., behave as no-operation instructions),
% and (3) a static control where the environment does not fluctuate.
% We compare the evolutionary dynamics of evolving populations during the [second] phase of the experiment.
% This ensures that we limit our observations to populations that are well-adapted to their experimental treatment.
% Because we only continue to the [second] phase in the plasticity-enabled condition if adaptive plasticity evolves, we limit our second phase measurements to phenotypically plastic populations.  



% Specifically, we compare the phenotypic volatility and mutation accumulation along successful lineages as well as the number of coalescence events that occur in plastic and non-plastic populations evolving in fluctuating environments.
% These measurements are described in further detail in Section [blah].





% @AML: why do we compare?

% We conducted a two-phase experiment to test whether adaptive phenotypic plasticity can dampen the rate of evolutionary change in fluctuating environments.
% The first, ``acclimation'' phase affords populations time to adapt to a given treatment; we limit our analyses to the second, ``[observation]'', phase of the experiment.
%  ; additionally, we compare the number of coalescence events that occur in evolving populations.
%  mutation accumulation, and frequency of selective sweeps in plastic and non-plastic populations evolving in a cyclically changing environment.

% During the first, ``acclimation'', phase of the experiment, we subject evolving populations of digital organisms to a fluctuating environment for 200,000 Avida updates%\footnote{
    %Updates in Avida are an experimental length of time. 
    %One update in Avida is the amount of time required for the average organism to execute 30 instructions. 
    %See [cite] for more details.
%}.
% 

% 

%%%%%%%%%%%%%%%%%%%%%%%%%%%%%%%%%%%%%%%%%%%%%%%%%%%%%%%%%%%%%%%%%%%%
% % In this work, we measure rates of evolutionary change in evolving populations using lineage metrics (as described by Dolson et al. in [cite]) and by tracking coalescence events. 
% A complete lineage describes a continuous line of descent, specifying an unbroken chain of parent-offspring relationships.
% %We isolated the lineage from the phase two ancestral seed genotype to the dominant genotype in the final population.
% We isolated the lineage for phase two of each replicate from its seed genotype to the dominant genotype at the end of its evolution.


% of the dominant genotypes at the end of phase two; 
% these lineages are bookended by an extant genotype and the original ancestral genotype used seed the population at the beginning of the observation phase. 

% @AML: can we better/more explicitly tie in how these represent evolutionary change (idk, maybe it's obvious)?
% We focus on two lineage metrics: mutation accumulation and phenotypic volatility.
% Mutation accumulation measures the total number of mutational changes along a lineage.
% For example, we might trivially expect [populations evolving/lineages evolved] under high mutation rates to exhibit high mutation accumulation [cite - tape of life].
% We measure phenotypic volatility as the sum of phenotypic changes (from parent to offspring) observed along a lineage [cite].
% We might expect elevated levels of both phenotypic volatility and mutation accumulation in populations evolving in a continuously changing environment where they must continuously adapt via \textit{de novo} genetic changes [cite - tape of life].

% coalescence events
% @AML: i feel like this paragraph is missing a little something
% In asexual populations without ecological interactions fostering coexistence, phylogenies should coalesce periodically (i.e., the most recent common ancestor shared by the extant population should change).
% The rate at which these coalescence events occur is driven by the strength of selection; populations under strong selection pressure should experience more rapid coalescence events than populations under weaker selection [cite?].
% For our experiments, we track the rate of coalescence events: how often the most recent common ancestor for each evolving population changes.
%%%%%%%%%%%%%%%%%%%%%%%%%%%%%%%%%%%%%%%%%%%%%%%%%%%%%%%%%%%%%%%%%%%%




\vspace{0.5cm}
\subsection{Measuring evolutionary change}
\label{sec:methods:measuring_evoluationray_change}

In this work, we measure rates of evolutionary change in evolving populations using lineage metrics (as described by Dolson et al. in [cite] and [cite]) and by tracking coalescence events. 
A complete lineage describes a continuous line of descent, specifying an unbroken chain of parent-offspring relationships.
For this work, we analyze the lineages of dominant (i.e., the most numerous) genotypes at the end of an experiment's [observation] phase; these lineages are bookended by an extant genotype and the original ancestral genotype used seed the population at the beginning of the observation phase. 

% @AML: can we better/more explicitly tie in how these represent evolutionary change (idk, maybe it's obvious)?
We focus on two lineage metrics: mutation accumulation and phenotypic volatility.
Mutation accumulation measures the total number of mutational changes along a lineage.
For example, we might trivially expect [populations evolving/lineages evolved] under high mutation rates to exhibit high mutation accumulation [cite - tape of life].
We measure phenotypic volatility as the sum of phenotypic changes (from parent to offspring) observed along a lineage [cite].
We might expect elevated levels of both phenotypic volatility and mutation accumulation in populations evolving in a continuously changing environment where they must continuously adapt via \textit{de novo} genetic changes [cite - tape of life].

% coalescence events
% @AML: i feel like this paragraph is missing a little something
In asexual populations without ecological interactions fostering coexistence, phylogenies should coalesce periodically (i.e., the most recent common ancestor shared by the extant population should change).
The rate at which these coalescence events occur is driven by the strength of selection; populations under strong selection pressure should experience more rapid coalescence events than populations under weaker selection [cite?].
For our experiments, we track the rate of coalescence events: how often the most recent common ancestor for each evolving population changes.



[Deleterous hit deleterious alleles/genes ``hitchhike'' to high frequencies within a population along with a beneficial allele/gene].
We repeated the experiment described in Section \ref{sec:methods:experiment-evolutionary-dynamics}; however, during the [second] phase of evolution, we introduced a purely deleterious instruction, \code{inst-poison}, that can mutate into offspring genomes.
When executed, \code{inst-poison} reduces an organism's reproductive success by decreasing its metabolic rate by [\instPoisonMagnitude].
Because executing \code{inst-poison} is always deleterious, we expect it to increase in prevalence primarily due to hitchhiking effects.
At the end of the experiment, we compare the number of \code{inst-poison} instructions executed by the final dominant genotypes across treatments.
Additionally, we analyze the prevalence of these deleterious instructions along the lineages of these genotypes.

%%%%%%%%%%%%%%%% old %%%%%%%%%%%%%%%%
We found no significant difference in the number of elapsed generations in populations across experimental conditions [stats/supplement].

Figure [X] gives the number of coalescence events, mutation accumulation, and phenotypic volatility for PLASTIC, NON-PLASTIC, and STATIC experimental treatments.
Across each of these metrics, we found that NON-PLASTIC populations experienced more evolutionary change than PLASTIC and STATIC populations.
We found that NON-PLASTIC populations underwent significantly more coalescence events than both PLASTIC and STATIC populations [stats], and NON-PLASTIC lineages had significantly higher mutation accumulation and phenotypic volatility than PLASTIC and STATIC lineages [stats].

The differences in mutation accumulation across conditions also manifest in the observed length of dominant genotypes; we observed significantly longer genomes in dominant organisms from NON-PLASTIC populations than in PLASTIC or STATIC populations [stats].
% very clunky sentence that might actually belong in discussion
We hypothesize that both the larger mutation accumulation and the larger genomes in NON-PLASTIC populations is primarily a result of neutral or mildly deleterious mutations hitchhiking to fixation along with beneficial mutations that genetically activate or deactivate a logic task [to better align with current environmental conditions].
% [if we include the above sentence, need a sentence or two about coalescence events and phenotypic volatility]


We observed that the evolutionary dynamics of plastic populations evolving in a fluctuating environment are more similar to that of non-plastic populations evolving in a constant environment than to non-plastic populations evolving in fluctuating environment.
We found no significant difference in number of coalescence events or in mutation accumulation across PLASTIC and STATIC treatments [stats].
% [mention relaxed selection?]
We did observe significantly higher phenotypic volatility in PLASTIC lineages than in STATIC lineages; this is likely due to the possibility for cryptic variation to accumulate in plastic genomes, which is temporarily hidden to selection (until the environment changes) but revealed in our measure of phenotypic volatility.




71 two- and three-input logic tasks (in addition to the treatment-specific environment) that organisms can perform to improve their metabolic rate. 
The full suite of all possible 77 one-, two-, and three-input logic tasks is a standard environment in Avida [cite]; see our supplemental material for a full list of additional logic tasks [cite].

% GENOME LENGTH RESULTS (braindump state)??
% The differences in mutation accumulation across conditions also manifest in the observed length of dominant genotypes; we observed significantly longer genomes in dominant organisms from NON-PLASTIC populations than in PLASTIC or STATIC populations [stats].
% We hypothesize that both the larger mutation accumulation and the larger genomes in NON-PLASTIC populations is primarily a result of neutral or mildly deleterious mutations hitchhiking to fixation along with beneficial mutations that genetically activate or deactivate a logic task [to better align with current environmental conditions].

% We independently evaluated a range of penalties imposed on organisms each time they executed a \code{poison} instruction: no penalty, a small penalty (3\% metabolic rate reduction), an intermediate penalty (10\% metabolic rate reduction), and a large penalty (30\% metabolic rate reduction).
% Without a metabolic rate penalty, executing \code{poison} is neutral; each other version of the \code{poison} instruction is explicitly deleterious.


% We experimentally manipulated the capacity for adaptive phenotypic plasticity in order to directly test how adaptive plasticity shifts evolutionary dynamics in a simple cyclic environment relative to non-plastic populations evolving under otherwise identical conditions. 

% In asexual populations without horizontal gene transfer, all mutations are linked.
% As such, neutral or deleterious mutations that co-occur with a beneficial mutation can sometimes `hitchhike' to fixation [cite - Van den Bergh 2018].
% Without co-occurrence, natural selection normally prevents deleterious mutations from reaching high frequencies, while genetic drift will allow some neutral mutations to eventually fix [citations]. 
% However, hitchhiking effects have been shown to drive both neutral [citations] and deleterious mutations to high frequencies [citations].



% - plasticity mediates how environmental variation influences population dynamics [reed et al., 2010]


% These genomes are built using a robust genetic language that can always be executed (though may not always be meaningful) and is expressive enough to produce any computation (i.e., it is Turing complete).

%%%%%%%%%
% Introduction - 2021-02-20
%%%%%%%%%

% Further, under the conventional perspective of adaptive evolution that focuses 
% selection that acts on non- heritable phenotypic variation in a population is often regarded as selection that does not produce an evolu- tionary response [ghalambor et al 2007]

% [Plasticity has been hypothesized to constrain evolutionary change because most of the phenotypic variation that results from plastic responses is environmentally induced and thus non-heritable, shielding other traits from selection \citep{gupta_study_1982,ancel_undermining_2000,huey_behavioral_2003,price_role_2003,paenke_influence_2007}.] % [citations]
% [Take more time/care with the plasticity constrains evolution perspective here].

% Why might plasticity slow evolution?
% - buffers population against environmental changes => individuals can rely on environmentally-induced phenotypic changes than evolving new traits. 

% Given genetic variation in the tendency to produce such variants, selection can refine these traits through genetic changes over time (i.e., genetic accommodation); further, if novel conditions persist, selection may drive plastic phenotypes to lose their environmental dependence over time in a process known as genetic assimilation \citep{west-eberhard_developmental_2005,pigliucci_phenotypic_2006,crispo_baldwin_2007,schlichting_phenotypic_2014,levis_evaluating_2016}. 
% Plastic rescue 

% These hypotheses are not mutually exclusive; the circumstances under which plasticity promotes or constrains further evolution will depend on many factors, such as [form of plasticity, genetic background, ...] \cite{ghalambor_non-adaptive_2015}.
% Understanding the contexts under which plasticity promotes or constrains further evolution is important if we are to predict and limit the damage caused by the acceleration of environmental shifts due to global climate change [citations].

% O: In digital evolution, self-replicating computer programs (digital organisms) compete for resources, mutate, and evolve \textit{in silico} \citep{mckinley_harnessing_2008}.
% NG: Digital evolution experiments where self-replicating computer programs (digital organisms) compete  for resources, mutate, and evolve \textit{in silico} following Darwinian dynamics have emerged as a powerful research framework from which evolution can be studied. 

% In asexual populations without horizontal gene transfer, all mutations are linked.
% As such, deleterious mutations that co-occur with a beneficial mutation can sometimes `hitchhike' to fixation \citep{van_den_bergh_experimental_2018}.
% Without such co-occurrence, natural selection normally prevents deleterious mutations from reaching high frequencies [citations]. 
% However, hitchhiking effects have been shown to drive both deleterious mutations to high frequencies [citations]. 

% genomic and phenotypic change, the capacity to evolve and then maintain novel traits, and the accumulation of deleterious alleles
% Our experiments build on previous digital evolution studies on the evolution of adaptive phenotypic plasticity \citep{clune_investigating_2007,lalejini_evolutionary_2016} and on the evolutionary consequences of fluctuating environments \citep{li_digital_2004,canino-koning_fluctuating_2019}.
% Specifically, we experimentally manipulated the capacity for adaptive phenotypic plasticity to evolve in a fluctuating environment to isolate plasticity's effects on evolutionary outcomes.

% - Plasticity stabilizing force => slow evolutionary change. Phenotypic variation in population is environmentally induced.
% - If so, expect that these metrics look similar for plastic populations in fluctuating environment and populations evolving in constant environment.
% - Expect non-plastic populations => rapid evolutionary change as populations continuously adapt to environmental fluctuations via de novo genetic changes.
% We evaluated the propensity for genetic hitchhiking across PLASTIC, NON-PLASTIC, and STATIC treatment populations.