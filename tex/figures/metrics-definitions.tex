%%%%%%%%% BOX %%%%%%%%%%%%
% this is my current "definition" environment
% \newtcolorbox[auto counter]{definitions}[1][] {
%   enhanced,
%   breakable,
%   arc=0mm,
%   title={\textbf{Box \thetcbcounter. Metrics}},
%   #1
% }

% \begin{definitions}[colback=blue!5!white,colframe=blue!75!black,label=box:metrics]

% Hello world.

% \end{definitions}
%%%%%%%%%%%%%%%%%%%%%%%%%%

% @AML: Not sure the best way to start the description of each metric. 

\newcommand{\SweepsMetricName}{
Coalescence event count
}
\newcommand{\SweepsMetricDesc}{
Number of coalescence events that have occurred, which indicates the frequency of selective sweeps in the population.
}

\newcommand{\MutationCountMetricName}{
Mutation count
}
\newcommand{\MutationCountMetricDesc}{
Sum of all mutations that have occurred along a lineage.
}

% Phenotypic volatility => 
% - 
\newcommand{\PhenotypicVolatilityMetricName}{
Phenotypic volatility
}
\newcommand{\PhenotypicVolatilityMetricDesc}{
Number of instances where parent and offspring phenotypic profiles do not match along a lineage.
% Phenotypic volatility as defined here indicates the rate at which accumulated genetic changes actually change the phenotype along a lineage.
}

\newcommand{\MutationalStabilityMetricName}{
Mutational robustness
% OLD: Mutational stability
}
\newcommand{\MutationalStabilityMetricDesc}{
Proportion of mutations (from the set of all possible one-step mutations) that do not change the phenotypic profile of a focal genotype. We also measured \textit{realized mutational robustness}, which is the proportion of mutated offspring along a lineage whose phenotypic profile matches that of their parent. 
% OLD DEFS:
% - Proportion of mutated offspring along a lineage whose phenotypic profile matches that of their parent. 
% - Proportion of mutants whose phenotypic profile matches that of their parent. Two variations were recorded: lineage mutational stability which examines the mutated offspring along a lineage, and neighborhood mutational stability which examines a set of possible mutations on a given genotype.
}


% Mutational stability => phenotypic stability / mutational robustness
% - Lineage => realized phenotypic stability / realized stability / realized mutational robustness
% - Neighborhood => potential phenotypic stability / potential stability / potential mutational robustness

% - Mutational robustness
% - Realized mutational robustness

\newcommand{\TaskPerformanceMetricName}{
Final novel task count
}
\newcommand{\TaskPerformanceMetricDesc}{
Count of unique novel tasks performed by the representative organism in a final population from experiment \hyperref[sec:methods:exp:novel-task-evolution]{phase 2B}. 
This metric can range from 0 to 71 and measures the level of exploitation of the fitness landscape (\textit{i.e.}, the mapping between genetic space and phenotype space) at a given point in time.
% We focused on an organism from the dominant genotype at the end of the experiment as the most representative phenotype in the evolved population.
% Final task count is equivalent to the ``task performance'' metric in \citep{canino-koning_fluctuating_2019}.
}

\newcommand{\TaskDiscoveryMetricName}{
Novel task discovery
}
\newcommand{\TaskDiscoveryMetricDesc}{
Number of unique novel tasks ever performed along a given lineage in experimental \hyperref[sec:methods:exp:novel-task-evolution]{phase 2B}, even if a task is later lost.
This metric can range from 0 to 71 and measures a given lineage's level of exploration of the fitness landscape.
}

\newcommand{\TaskLossMetricName}{
Novel task loss
}
\newcommand{\TaskLossMetricDesc}{
Number of instances along a given lineage from experimental \hyperref[sec:methods:exp:novel-task-evolution]{phase 2B} where a novel task is performed by a parent but not its offspring. 
This metric measures how often a given lineage fails to retain evolved traits over time.
}

\newcommand{\FinalPoisonMetricName}{
Final poisonous task count
}
\newcommand{\FinalPoisonMetricDesc}{
Number of times the poisonous task is performed by the representative organism from a final population from experiment \hyperref[sec:methods:exp:deleterious-instruction-accumulation]{phase 2C}.
}

\newcommand{\LineagePoisonMetricName}{
Poisonous task acquisition count
}
\newcommand{\LineagePoisonMetricDesc}{
Number of instances along a given lineage where a mutation causes an offspring to perform the poisonous task more times than its parent. 
}

\newcommand{\ArchitectureVolatilityMetricName}{
Architectural volatility
}
\newcommand{\ArchitectureVolatilityMetricDesc}{
The average number of loci in the genome that change function per mutation along a lineage. 
% @AML: The average number of mutations that cause a change in function along a lineage. 
%Loci function is denoted as the combination of task machinery, plasticity machinery, vestigial machinery for both ENV-A and ENV-B tasks, as well as replication and required machinery. 
}

\newcommand*{\thead}[1]{\multicolumn{1}{c}{\bfseries #1}}

\setlength{\tabcolsep}{16pt}
\renewcommand{\arraystretch}{1.5}
\begin{table}[ht]
    \centering
    
    \rowcolors{2}{gray!25}{white}
    \begin{tabularx}{\linewidth}{lX} % p{10cm}
        \rowcolor{gray!50}
        \hline 
        \thead{Metric} & \thead{Description}   \\
        \hline
        \SweepsMetricName & \SweepsMetricDesc \\
        \MutationCountMetricName & \MutationCountMetricDesc \\
        \PhenotypicVolatilityMetricName & \PhenotypicVolatilityMetricDesc \\
        % \GenotypicFidelityMetricName &         \GenotypicFidelityMetricDesc \\
        % \PhenotypicFidelityMetricName &         \PhenotypicFidelityMetricDesc \\
        \MutationalStabilityMetricName & \MutationalStabilityMetricDesc \\
        % (architecture metric(s)) & \\
        %\ArchitectureVolatilityMetricName & \ArchitectureVolatilityMetricDesc \\
        \TaskPerformanceMetricName & \TaskPerformanceMetricDesc \\
        \TaskDiscoveryMetricName & \TaskDiscoveryMetricDesc \\
        \TaskLossMetricName & \TaskLossMetricDesc \\
        \FinalPoisonMetricName & \FinalPoisonMetricDesc \\
        \LineagePoisonMetricName & \LineagePoisonMetricDesc \\
        \hline
    \end{tabularx}
    
    \caption{\textbf{Metric descriptions.}}
    \label{tab:metrics-definitions}
\end{table}


