% GUIDELINES:
% This section may be divided by subheadings. Discussions should cover the key findings of the study: discuss any prior research related to the subject to place the novelty of the discovery in the appropriate context, discuss the potential shortcomings and limitations on their interpretations, discuss their integration into the current understanding of the problem and how this advances the current views, speculate on the future direction of the research, and freely postulate theories that could be tested in the future.

\section{Discussion}

In this work, we used evolving populations of digital organisms to determine how adaptive phenotypic plasticity alters subsequent evolutionary dynamics and influences evolutionary outcomes in fluctuating environments.
Specifically, we compared lineages of adaptively plastic organisms in fluctuating environments to both non-plastic organisms in those same environments and other non-plastic organisms in static environments.

\subsection{Evolutionary change}

% -- Adaptively plastic populations underwent less evolutionary change than non-plastic populations --
We found strong evidence that adaptive plasticity slows evolutionary change in fluctuating environments. 
Adaptively plastic populations experienced fewer coalescence events and fewer total genetic changes relative to non-plastic populations evolving under identical environmental conditions (Figure \ref{fig:evolutionary-dynamics-magnitude}).
Whereas non-plastic populations relied on \textit{de novo} mutations to adapt to each environmental fluctuation, plastic populations leveraged sensory instructions to regulate task performance. 
Indeed, in fluctuating environments, selection pressures toggle after each environmental change.
We hypothesize that in non-plastic populations such toggling would repeatedly drive the fixation of mutations that align an organism's phenotypic profile to the new conditions.
This hypothesis is supported by the increased frequency of coalescence events in these populations (Figure \ref{fig:evolutionary-dynamics-rate}a) as well as increased rates of genetic and phenotypic changes observed along the lineages of non-plastic organisms. 


%TODO: Make sure we use org vs. genotype consistently with rest of paper
% - Mutational neighborhood results
%We expected this was [caused by / due to / the result of] non-plastic populations evolving a bet-hedging strategy where mutations are more likely to modify the phenotypic profile.
%However, when we repeated the mutational robustness analysis on the mutational neighborhoods of representative genotypes (\textit{i.e.}, all possible one-step mutants), across the three conditions we observed the highest mutational robustness in the non-plastic condition (Figure \ref{fig:neighborhood-mutational-stability}).
Representative lineages in the non-plastic treatment experienced lower realized mutational robustness than plastic and static lineages (Figure \ref{fig:evolutionary-dynamics-rate}b).
We reasoned that this lower realized mutational robustness was due to non-plastic populations evolving a bet-hedging strategy where mutations are more likely to modify the phenotypic profile. 
However, when we switched from measuring the realized mutational robustness of representative lineages to measuring the mutational robustness of representative genotypes (i.e., what fraction of one-step mutants change the phenotypic profile), we observed that non-plastic genotypes exhibited the highest mutational robustness of all three treatments (Figure \ref{fig:mutational-robustness}).
This result runs contrary to both our expectations and the results of other fluctuating environment studies in Avida \citep{canino-koning_fluctuating_2019}.
\cite{canino-koning_fluctuating_2019} found that mutational robustness is negatively correlated with the number of task-encoding sites in the genome.
In our work, most plastic and static genotypes encode all six base tasks, while most non-plastic genotypes only encode tasks from one environment; this results in fewer task-encoding sites and thus may increase mutational robustness in non-plastic genotypes. %relative to genotypes from other treatments. 
Regardless of the cause, this higher mutational robustness in non-plastic organisms indicates that bet-hedging is not driving the lower realized mutational robustness observed in non-plastic lineages.
Thus, we expect the lower realized mutational robustness in non-plastic lineages to be driven by survivorship bias. 
%Because non-plastic lineages must rely on mutations to adapt to environmental changes, these adaptive mutations are highly beneficial and are thus selected. 
%This decreases the realized mutational robustness of the lineage even though
Because non-plastic lineages must rely on mutations to adapt to environmental changes, phenotype-altering mutations are often highly advantageous, and their selection decreases the realized mutational robustness of the lineage. 
%We expect this result to thus be due to survivorship bias, as non-plastic lineages rely on mutations to adapt to their environment, and thus the phenotype-changing mutations that do occur are highly adaptive and thus selected. 
%Future work is needed to determine if there are other contributing factors. 


% -- in context of previous digital evolution work --
To our knowledge, this study is the first in-depth empirical investigation into how the \textit{de novo} evolution of adaptive plasticity shifts the course of subsequent evolution in a cyclic environment.
The relative rates of evolutionary change that we observed in non-plastic populations, however, are consistent with results from previous digital evolution studies. 
For example, \cite{dolson_interpreting_2020} showed that non-plastic populations that were evolved in cyclically changing environments exhibited higher phenotypic volatility and accumulated more mutations than that of populations evolved under static conditions.
Furthermore, \cite{lalejini_evolutionary_2016} visually inspected the evolutionary histories of non-plastic organisms evolved in fluctuating environments, observing that mutations along successful lineages readily switched the set of traits expressed by offspring.
%\cite{canino-koning_evolution_2016} also observed that genomes evolved in harsh cyclic environments often contained vestigial fragments of genetic material adapted to prior environments.


% -- In context of conventional evolutionary theory: evo response => f(selection, variation) --
Our results are also consistent with conventional evolutionary theory.
A trait's evolutionary response to selection depends on the strength of directional selection and on the amount of genetic variation for selection to act upon \citep{lande_measurement_1983,zimmer_evolution_2013}.
In our experiments, non-plastic populations repeatedly experienced strong directional selection to toggle which tasks were expressed after each environmental change.
As such, retrospective analyses of successful lineages revealed rapid evolutionary responses (that is, high rates of genetic and phenotypic changes).
Evolved adaptive plasticity shielded populations from strong directional selection when the environment changed by eliminating the need for a rapid evolutionary response to toggle task expression.
Indeed, both theoretical and empirical studies have shown that adaptive plasticity can constrain evolutionary change by weakening directional selection on evolving populations \citep{price_role_2003,paenke_influence_2007,ghalambor_non-adaptive_2015}. 

% TODO, consider: Expectations that could arise if we took into consideration lineages off the line of descent. 

% \vspace{0.25cm}
\subsection{The evolution and maintenance of novel tasks}

% -- Exploration + Exploitation --
In fluctuating environments, non-plastic populations explored a larger area of the fitness landscape than adaptively plastic populations (Figure \ref{fig:complex-traits-magnitude}b).
However, adaptively plastic populations better exploited the fitness landscape, retaining a greater number of novel tasks than non-plastic populations evolving under identical environmental conditions (Figure \ref{fig:complex-traits-magnitude}a).
In our experiment, novel tasks were less important to survival than the fluctuating base tasks.
In non-plastic populations, when a mutation changes a base task to better align with current environmental conditions, its benefit will often outweigh the cost of losing one or more novel tasks. 
%Indeed, we found that mutations associated with novel task loss along representative lineages from non-plastic populations co-occurred with phenotypic changes that helped offspring adapt to current environmental conditions 97\% of the time.
Indeed, we found that along non-plastic representative lineages, 97\% of the mutations associated with novel task loss co-occurred with phenotypic changes that helped offspring adapt to current environmental conditions. 

% -- Changing environments promote evolutionary change --
Previous studies have shown that transitory environmental changes can improve overall fitness landscape exploration in evolving populations of non-plastic digital organisms \citep{nahum_improved_2017}.
Similarly, changing environments have been shown to increase the rate of evolutionary adaptation in simulated network models \citep{kashtan2007varying}.
In our system, however, we found that \textit{repeated} fluctuations reduced the ability of non-plastic populations to maintain and exploit tasks; that said, we did find that repeated fluctuations may improve overall task discovery by increasing generational turnover. 
Consistent with our findings, \cite{canino-koning_fluctuating_2019} found that non-plastic populations of digital organisms evolving in a cyclic environment maintained fewer novel traits than populations evolving in static environments.

% -- Plastic rescue, stabilizing effect of plasticity --
Our results suggest that adaptive phenotypic plasticity can improve the potential for populations to exploit novel resources by stabilizing them against stressful environmental changes.
The stability that we observe may also lend some support to the hypothesis that phenotypic plasticity can rescue populations from extinction under changing environmental conditions \citep{chevin_adaptation_2010}.

% -- relevance to genes as followers hypothesis --.
Our data do not necessarily provide evidence for or against the genes as followers hypothesis.
The genes as followers hypothesis focuses on contexts where plastic populations experience novel or abnormally stressful environmental change.
However, in our system, environmental changes were cyclic (not novel), and changes were not \textit{abnormally} stressful.
Further, the introduction of novel tasks during the second phase of the experiment merely added static opportunities for fitness improvement.
This addition did not change the meaning of existing environmental cues, nor did it require those cues to be used in new ways. 
 

% \vspace{0.25cm}
\subsection{The accumulation of deleterious alleles}

% -- Overview of results --
%   - More accumulation in non-plastic
%   - no evidence for cryptic variation housing poison
%   - plastic ~~ static
We found that non-plastic lineages that evolved in a fluctuating environment exhibited both greater totals and higher rates of poisonous task acquisition than that of adaptively plastic lineages (Figure \ref{fig:deleterious-hitchhiking}).
In asexual populations without horizontal gene transfer, all co-occurring mutations are linked.
As such, deleterious mutations linked with a stronger beneficial mutation (\textit{i.e.}, a driver) can sometimes ``hitchhike'' to fixation \citep{smith_hitch-hiking_1974,van_den_bergh_experimental_2018,buskirk_hitchhiking_2017}.
Natural selection normally prevents deleterious mutations from reaching high frequencies, as such mutants are outcompeted.
However, when a beneficial mutation sweeps to fixation in a clonal population, it carries along any linked genetic material, including other beneficial, neutral, or deleterious mutations \citep{barton_genetic_2000, smith_hitch-hiking_1974}.
Therefore, we hypothesize that deleterious genetic hitchhiking drove \code{poison} instruction accumulation along non-plastic lineages in changing environments.

Across our experiments, the frequency of selective sweeps in non-plastic populations provided additional opportunities for genetic hitchhiking with each environmental change. 
Indeed, representative lineages from non-plastic populations in the cyclic environment exhibited higher mutation accumulation (Figure~\ref{fig:evolutionary-dynamics-magnitude}b), novel trait loss (Figure~\ref{fig:complex-traits-magnitude}c), and poisonous task acquisition (Figure~\ref{fig:deleterious-hitchhiking}a) than their plastic counterparts.
In aggregate, we found that many ($\sim$49\%; 956 / 1916) mutations that increased \code{poison} instruction execution in offspring co-occurred with mutations that provided an even stronger benefit by adapting the offspring to an environmental change.
We expect that an even larger fraction of these deleterious mutations were linked to beneficial mutations, but our analysis only counted mutations that co-occurred in the same generation.


% -- Elaboration on plastic result --
Theory predicts that under relaxed selection deleterious mutations should accumulate as cryptic variation in unexpressed traits \citep{lahti_relaxed_2009}.
Contrary to this expectation, we did not find evidence of \code{poison} instructions accumulating as cryptic variation in adaptively plastic lineages.
One possible explanation is that the period of time between environmental changes was too brief for variants carrying unexpressed \code{poison} instructions to drift to high frequencies before the environment changed, after which purifying selection would have removed such variants.
Indeed, we would not expect drift to fix an unexpressed trait since we tuned the frequency of environmental fluctuations to prevent valuable traits from being randomly eliminated during the off environment.
Additionally, plastic organisms in Avida usually adjust their phenotype by toggling the expression of a minimal number of key instructions, leaving little genomic space for cryptic variation to accumulate.

\subsection{Limitations and future directions}

% ------------
% Possible additional future directions:
% - Need to look at mutations off the line of descent 
% - Variable length genomes
% - Relaxed selection, mutational decay
% ------------

% -- Adaptive vs non-adaptive plasticity --
Our work lays the groundwork for using digital evolution experiments to investigate the evolutionary consequences of phenotypic plasticity in a range of contexts.
However, the data presented here are limited to the evolution of adaptively plastic populations.
Future work might explore the evolutionary consequences of maladaptive and non-adaptive phenotypic plasticity (\textit{e.g.}, \citealt{leroi_temperature_1994}), which are known to bias evolutionary outcomes \citep{ghalambor_non-adaptive_2015}. 
% -- Environmental change --
Additionally in our experiments, sensory instructions perfectly differentiated between ENV-A and ENV-B, and environmental fluctuations never exposed populations to entirely new conditions.
These parameters have been shown to influence evolutionary outcomes \citep{li_digital_2004,boyer_adaptation_2021}, which if relaxed in context of further digital evolution experiments, may yield additional insights.

% - limitation: focus on lineages -
%   - extend to complete evolutionary histories
We focused our analyses on the lineages of organisms with the most abundant genotype in the final population.
These successful lineages represented the majority of the evolutionary histories of populations at the end of our experiment, as populations did not exhibit long-term coexistence of different clades.
Our analyses, therefore, gave us an accurate picture of what fixed in the population.
We did not, however, examine the lineages of extinct clades.
Future work will extend our analyses to include extinct lineages, giving us a more complete view of evolutionary history, which may allow us to better distinguish adaptively plastic populations from populations evolving in a static environment. 

% - Machinery -
As with any wet-lab experiment, our results are in the context of a particular model organism: ``Avidian'' self-replicating computer programs.
Digital organisms in Avida regulate responses to environmental cues using a combination of sensory instructions and conditional logic instructions (\code{if} statements).
The \code{if} instructions conditionally execute a single instruction depending on previous computations and the state of memory. 
As such, plastic organisms in Avida typically regulate phenotypes by toggling the expression of a small number of key instructions as opposed to regulating cohorts of instructions under the control of a single regulatory sequence \citep{supplemental_material}. 
This bias may limit the accumulation of hidden genetic variation in Avida genomes. 
However, as there are many model biological organisms, there are many model digital organisms that have different regulatory mechanisms that should be used to test the generality of our results.

% [A broad sentence to wrap everything up?]
% e.g., A sentence about how we hope this work inspires more use of digital evolution as an experimental tool/expand experimental repertoire of evolutionary biologists studying phenotypic plasticity?
% e.g., As demonstrated here, Digital evolution studies allow us to directly manipulate the capacity for plastic responses to evolve and perfectly observe subsequent dynamics, enabling us to experimentally test hypotheses that were previously relegated to theoretical analyses.

